\section{Introduction}
\setlength{\parindent}{5ex}
Dans le cadre du projet de S2 à EPITA, nous sommes poussés à mettre en pratique les différentes connaissances acquises en Travaux Pratiques et en cours, dans la réalisation d'un projet en groupe.
Le projet de notre studio gameHUB est la création d'un jeu d'horreur: \emph{Nyctalopia}.


À l'arrivée de la première soutenance, nous avons préféré nous concentrer sur l'aspect technique du jeu, le ``backend'', toute la partie que le ou les joueurs ne verraient pas, ce qui nous a permis pour cette deuxième date clé de notre projet, d'avancer plus rapidement afin d'obtenir des résultats plus visibles surtout dans les parties graphiques et jouables.

Pour la seconde soutenance, ayant eu moins de temps, par rapport à la première soutenance, nous avions décidé de régler les bogues pouvant être très gênants pour l'utilisateur, ainsi qu'implémenter de quelques fonctionnalités au niveau du Gameplay.

Dans cette dernière ligne droite, nous avons essayé de clôturer notre projet en intégrant toutes les parties manquantes du jeu, notamment le nouveau \emph{Chapitre 3: Les Égouts}. Même si tous nos objectifs n'ont pas pu être atteints, nous avons voulu présenter une version finale propre et fonctionnelle.

Nous présentons, dans ce rapport de projet, tout le progrès réalisé dans notre projet lors de ce semestre dans la réalisation de notre jeu-vidéo d'horreur: \emph{Nyctalopia}.

\vfill
\noindent\makebox[\linewidth]{\rule{.8\paperwidth}{.6pt}}\\[0.2cm]
EPITA Toulouse - Projet S2 - 2022 \hfill Nyctalopia - gameHUB
\noindent\makebox[\linewidth]{\rule{.8\paperwidth}{.6pt}}

\newpage