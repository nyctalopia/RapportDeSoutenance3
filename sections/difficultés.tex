\section{Difficultées}
\setlength{\parindent}{5ex}
Lors du développement de ce jeu, nous avons rencontré divers problèmes qui nous on déranger dans notre travail et réduit notre efficacité.

\par
Le plus dérangeant de tout est le fait que nous avons décidé de faire un jeu en 3D avec des textures de bonne qualité. En effet, cela a rendu le jeu très lourd, au final nous avons réussi à réduire le taille du projet à environ 39 GB, mais lors du développement cela à empêché certaines personnes de pouvoir directement développer et tester sur le jeu causant des conflits entre les différentes implémentations. De plus, tout le groupe a passé un temps non-négligeable dans les écrans de chargement sur \emph{Unity3D}, notamment lors de la compilation des scripts et du lancement des scènes. Des écrans de chargements dont la durée a été accentuée par le manque de puissance de nos ordinateurs, souvent portables.

\vspace*{7mm}
\par
De plus, concernant les sauvegardes, malgré le fait que les scripts et les classes soient finalisés, nous avons rencontré un problème concernant l'interface. En effet, lors de l'apparition de l'interface des sauvegardes, qui était censée afficher les trois emplacements de sauvegarde, nous n'avons pas réussit à garder cette interface au milieu de l'écran du joueur rendant sont utilisation impossible car celle-ci était partiellement visible à l'écran.
Nous n'avons, par conséquent, pas pu implémenter celle-ci dans cette version du jeu et donc malgré nos efforts, le système de sauvegardes n'est pas fonctionnel.

\vspace*{7mm}
\par
Beaucoup de changements nous ont provoqué des ralentissements tout au long du projet, notamment le multijoueur, dont le SDK a été modifié plusieurs fois pour avoir un multijoueur compatible avec notre jeu et avec Steam.

\vspace*{7mm}
\par
L'intelligence artificielle et l'algorithme de pathfinding nous ont aussi posés des problèmes qu'on a finalement su résoudre au détriment d'un système d'attaque, de mort et de réapparition fonctionnels. Notre créature suivra parfaitement le joueur dans le chapitre 3 en traçant le chemin le plus court pour l'atteindre, mais malheureusement celle-ci ne sera pas capable d'attaquer celui-ci.


\vspace*{7mm}
\par
Au commencement du projet, nous avions l'ambition d'implémenter quelques caractéristiques pouvant rendre l'expérience du joueur plus agréable tel que la personnalisation des touches. Malheureusement, nous nous sommes rendu compte, peut être trop tard, que cela n'était pas vital au fonctionnement du jeu et nous nous sommes plutôt concentrés sur les bases pour avoir un jeu fonctionnel.

\vfill
\noindent\makebox[\linewidth]{\rule{.8\paperwidth}{.6pt}}\\[0.2cm]
EPITA Toulouse - Projet S2 - 2022 \hfill Nyctalopia - gameHUB
\noindent\makebox[\linewidth]{\rule{.8\paperwidth}{.6pt}}